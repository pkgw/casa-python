%% Copyright (C) 1999,2000,2001,2002,2003
%% Associated Universities, Inc. Washington DC, USA.
%%
%% This library is free software; you can redistribute it and/or modify it
%% under the terms of the GNU Library General Public License as published by
%% the Free Software Foundation; either version 2 of the License, or (at your
%% option) any later version.
%%
%% This library is distributed in the hope that it will be useful, but WITHOUT
%% ANY WARRANTY; without even the implied warranty of MERCHANTABILITY or
%% FITNESS FOR A PARTICULAR PURPOSE.  See the GNU Library General Public
%% License for more details.
%%
%% You should have received a copy of the GNU Library General Public License
%% along with this library; if not, write to the Free Software Foundation,
%% Inc., 675 Massachusetts Ave, Cambridge, MA 02139, USA.
%%
%% Correspondence concerning AIPS++ should be addressed as follows:
%%        Internet email: aips2-request@nrao.edu.
%%        Postal address: AIPS++ Project Office
%%                        National Radio Astronomy Observatory
%%                        520 Edgemont Road
%%                        Charlottesville, VA 22903-2475 USA
%%
%% $Id$
\providecommand{\briggsURL}{http://www.aoc.nrao.edu/ftp/dissertations/dbriggs/diss.html}
\providecommand{\pixonURL}{http://www.pixon.com/}
\providecommand{\rsiURL}{http://www.rsinc.com/}

\begin{ahmodule}{imager}{Module for synthesis and single dish imaging}
\ahinclude{imager.g}

\begin{ahdescription} 
\end{ahdescription}

{\tt imager} provides a unified interface for synthesis and single
dish imaging including deconvolution starting from a MeasurementSet.

\subsubsection*{What \texttt{imager} does:}

\begin{description}
\protect\item[Standard synthesis and single dish imaging] {\tt imager} does
nearly all types of synthesis and single dish imaging, including dirty
images, point spread functions, deconvolution, combination of single
dish and synthesis, spectral imaging, polarimetry, wide-field imaging,
mosaicing, holography, near-field imaging, tracking moving objects,
{\em etc.}. As a result of this extensive range of capabilities, it
can be complicated to use, especially for the more esoteric forms of
imaging.
\protect\item[Fine scaled tools] Rather than present one operation
to process data from visibilities to a restored, deconvolved
image, {\tt imager} contains a number of distinct, separate tool functions
that allow careful tuning of the processing. For example, the
weights used in imaging (the IMAGING\_WEIGHT column in the
MeasurementSet), can be altered via a number of tool functions
(\protect\ahlink{weight}{imager:imager.weight}, \protect\ahlink{filter}{imager:imager.filter}) 
and inspected via a plotting
tool function \protect\ahlink{plotweights}{imager:imager.plotweights}. Similarly, the
deconvolution and restoration steps are separate, allowing user
control of each step. It is our intention that other imaging tools may
be built on top of imager: see, for example,
\protect\ahlink{imagerwizard}{imager:imagerwizard}, which also has the side-benefit
that
it displays the imager (and other tools) commands as they are
executed.
\protect\item[Spectral imaging] \texttt{imager} can perform either spectral
imaging or frequency synthesis (producing either an image with 
each channel imaged independently or with some or 
all channels summed together).
Channels may be selected in a number of ways, either as channels
or as velocities. Also a continuum model image may be subtracted 
prior to making a cube.
\protect
\item[Many different deconvolution algorithms] {\tt imager} is rich in
deconvolution algorithms, including a number of clean variants,
maximum entropy, non-negative least squares, and the pixon algorithm.
\item[Mixing of deconvolution functions] Since the deconvolved images
are calculated and kept purely as images (rather than lists of clean
components), deconvolution functions may be mixed as desired. Thus,
one may use NNLS to deconvolve part of the Stokes I of an image, 
and then use CLEAN to deconvolve another part of all polarizations
in the image. Note that a list of clean components is not available.
\item[Ability to fix model images] In a multifield deconvolution,
it is possible to specify that some fields are not to be deconvolved,
using the {\tt fixed} argument of \ahlink{clean}{imager:imager.clean}.
\item[Single dish imaging] {\tt imager} can process single dish
observations much as it does synthesis images. To make images with no
deconvolution, use the \ahlink{makeimage}{imager:imager.makeimage}
function. This allows construction of traditional single dish images
and holography images. To deconvolve images, just use the
``multifield'' deconvolution algorithms in
\ahlink{clean}{imager:imager.clean} and 
\ahlink{mem}{imager:imager.mem}. You will want to set the gridmachine
in \ahlink{setoptions}{imager:imager.setoptions} to 'sd'.
\item[Combination of single dish and synthesis data] If the single
dish and interferometer data are in the same MeasurementSet, then
imager can perform a joint deconvolution using
``multifield'' deconvolution algorithms in
\ahlink{clean}{imager:imager.clean} and 
\ahlink{mem}{imager:imager.mem}. You will want to set the gridmachine
in \ahlink{setoptions}{imager:imager.setoptions} to 'both'. You can
change the relative weighting of synthesis and single dish data by
using \ahlink{setsdoptions}{imager:imager.setsdoptions}.
If the single dish and synthesis data cannot be combined into
one MeasurementSet then you can still use the
\ahlink{feather}{imager:imager.feather} function to combine already
deconvolved images.
\item[Multi-field processing] {\tt imager} can be run on any number
of images, each of which can have any direction for the phase
center. All coordinate transformations are done correctly.  Using the
measures system, these fields may be given moving positions (such as
the Sun using {\tt dm.direction('sun')} to specify the phase center) or
positions in strange coordinates (such as Supergalactic using
{\em e.g.} {\tt dm.direction('supergal', '0d', '0d')} as well as
the more conventional representations ({\em e.g.}
{\tt dm.direction('b1950', '12h26m33.248000', '02d19m43.290000')} specifies
the coordinates of the core of 3C273). Note that for
some coordinate systems a location must be supplied via
the \ahlink{setoptions}{imager:imager.setoptions} tool function. For example,
one can put an image at a specific azimuth and elevation ({\em e.g.}
{\tt dm.direction('azel', '67.4d', '5.23d')}) at the
VLA {\tt imgr.setoptions(location=dm.observatory('VLA'))}. Phase 
rotation will be automatically calculated to track in 
azimuth-elevation.
\item[Wide-field imaging] {\tt imager} can perform wide-field
imaging as needed to overcome the non-coplanar baselines
effect for the VLA and other non-coplanar arrays.
\item[Mosaicing] {\tt imager} can perform clean-based or mem-based mosaicing
of many pointings into one image, using variants of the multi-field
algorithms.
\item[Processing of component lists] Discrete components
(not the same as clean components!) can be represented by 
\ahlink{componentmodels}{componentmodels}. A componentlist
can hold any number of components. The components are subtracted from
the visibility data before construction of an image. For high
precision imaging, it is recommended that components be used for
bright sources since the Fourier Transform of components avoids the
limitations of the gridded transforms.
%\item[Self-calibration] Self-calibration is accomplished
%using the \ahlink{calibrater}{calibrater} tool. One sets up a calibrater tool
%with the necessary parameters and then passes it by name to
%the \ahlink{selfcal}{imager:imager.selfcal} tool function.
\item[Joint deconvolution of Stokes IQUV] {\tt imager} can produce
images of either $I$ alone, or $I,V$ or $I,Q,U,V$, deconvolving
jointly as appropriate. The point spread function is constrained to
be the same for all processed polarizations so asymmetric u,v coverage is
not allowed.
\item[Production of complex images] {\tt imager} can produce
dirty or residual images or point spread functions in the original 
data representation ({\em e.g.} RR,RL,RL,LL or XX,XY,YX,YY. 
\item[Fine control and evaluation of visibility weighting] Various
tool functions for controlling the visibility weights are available
(\ahlink{weight}{imager:imager.weight}, \ahlink{filter}{imager:imager.filter}) 
as well as tool functions for evaluating the effects of the weighting
(\ahlink{plotweights}{imager:imager.plotweights},
\ahlink{sensitivity}{imager:imager.sensitivity},
\ahlink{fitpsf}{imager:imager.fitpsf}).  The Briggs algorithm for 
weighting of visibility data can be used (see
\ahlink{weight}{imager:imager.weight} and \htmladdnormallink{Dan Briggs'
thesis} {\briggsURL}).
\item[Flexible windowing in the deconvolution] Rather than use boxes
to limit the region CLEANed, a mask image is used to constrain the
region in which flux is allowed. There are various tool functions for making
a mask image, including from regions and blc/trc specifications, and via
thresholding the Stokes I image. In the Clark Clean, the mask is
{\em soft}: it can vary between 0 and 1. Intermediate values of
the mask bias against but do not rule out subtraction of clean components.
\item[Non-Negative Least Squares Deconvolution] This algorithm
is very effective at producing high dynamic range images of moderately
resolved sources (see \htmladdnormallink{Dan Briggs' thesis}
{\briggsURL}). It works on Stokes $I$ alone so the recommended
procedure is to CLEAN $I,Q,U,V$ using \ahlink{clean}{imager:imager.clean}
and then use NNLS to refine the $I$ part of the image using
\ahlink{nnls}{imager:imager.nnls}.
%\item[Specification of arguments as \ahlink{measures}{measures} ]
\item[Specification of arguments as] A measure 
is a measured quantity with optional units, coordinates and reference
frames. These are allowed in a number of circumstances.  The advantage
is that the user can specify arguments in very convenient form, and
let the measures system do whatever conversion is required. For
example:
\begin{description}
\item[Cell sizes] These can be specified as a quantity (see the 
\ahlink{measures}{measures} module).
\begin{verbatim}
imgr.setimage(cellx='7arcsec', celly='7arcsec')
\end{verbatim}
\item[Image center direction] This must be specified as a 
direction (see the \ahlink{measures}{measures} module).
\begin{verbatim}
imgr.setimage(phasecenter=dm.direction('j2000', '05h30m', '-30.2deg'))
imgr.setimage(phasecenter=dm.direction('gal', '0deg', '0deg'))
imgr.setimage(phasecenter=dm.direction('mars'))
imgr.setimage(phasecenter=image('myother.image').coordmeasures().direction);
\end{verbatim}
\item[Velocities] These can be specified as radial velocities.
\begin{verbatim}
imgr.setimage(start=dm.radialvelocity('25km/s'), 
            step=dm.radialvelocity('-500m/s'))
\end{verbatim}
\item[Position] For construction of images in some coordinate
frames ({\em e.g} azimuth-elevation) the position to be used
in processing must be set:
\begin{verbatim}
imgr.setoptions(location=dm.observatory('ATCA'))
\end{verbatim}
\end{description}
\item[More choice in image size] Any even image size will work,
though to speed the FFT, it is advisable to
use a highly composite number (one that has many factors).
The \ahlink{advise}{imager:imager.advise} function will
calculate an acceptable number.
\item[Integrated plotting] Plots of visibility 
amplitude, weights (both point-by-point and gridded),
uv coverage, and field and spectral window ids are available 
(\ahlink{plotvis}{imager:imager.plotvis}, 
\ahlink{plotweights}{imager:imager.plotweights}, 
\ahlink{plotuv}{imager:imager.plotuv}, 
\ahlink{plotsummary}{imager:imager.plotsummary}).
\item[Synchronous or Asynchronous processing] Operations that
take a substantial amount of time to run can be run in the
background either by setting the global variable {\tt dowait:=F}
or by setting an argument {\em e.g.} {\tt imgr.clean(async=T)}.
To retrieve a result, use the result tool function of
defaultservers with the job number as the argument. For example:
\begin{verbatim}
- imgr:=imager('ss433.MS')
T
- imgr.setimage(cellx='0.05arcsec', celly='50marcsec', nx=256, ny=256, 
  spwid=1:2, fieldid=1, stokes='IV') 
T 
- imgr.fitpsf()
1 
# Wait for it to finish and then ask for the result:
- defaultservers.result(1)
[psf=, bpa=[value=42.7269936, unit=deg], bmin=[value=0.13008301, 
unit=arcsec], bmaj=[value=0.159367442, unit=arcsec]] 
\end{verbatim}
\item[A novel sort-less gridding algorithm] The visibility data are
not sorted before the gridding step. Instead, a cache of tiles is
allocated to hold each baseline as it moves around in the Fourier
plane. When a baseline moves off an existing tile, the results are
written to disk and the necessary new tile is read in. Since the
rotation of baselines in the uv plane is usually quite slow, the hit
rate of such a cache is high. The size of the cache is by default
set to half the physical memory of the machine, as specified
by the aipsrc variable system.resources.memory. This can be overridden
by the user, via the \ahlink{setoptions}{imager:imager.setoptions}
tool function. The cache
can be made smaller at the expense of more paging of tiles in and
out. The tile size can also be changed but this is seldom needed.
This approach is optimal for arrays with small numbers of antennas
but can be slow for {\em e.g.} the VLA. We intend to rectify this
in the near future.
\item[Plug-in commands] {\tt imager} can be customized by attaching
commands using the \aipspp\ plug-in system. See the file
code/trial/apps/imager/imager\_standard.gp for an example of how to 
attach commands. 
\item[Suite of tests] {\tt imager} has a suite of tests. A standard 
test data set and component list can also be created.
\item[imagerwizard] The \ahlink{simplemage}{imager:imagerwizard}
function is a wizard that performs interactively guided imaging
of synthesis data.
\item[dragon] The \ahlink{dragon}{imager:dragon} tool performs 
wide-field imaging using imager.
\item[vpmanager] The \ahlink{vpmanager}{imager:vpmanager} tool manages
specification of primary beams for imager.
\item[Near-field imaging {\em experimental}] Images of objects in the near-field of an
array can be made. If the distance to the object is specified 
in \ahlink{setimage}{imager:imager.setimage}, then the extra delay
due to the wavefront curvature is corrected in the transforms.
Note that some telescopes ({\em e.g.} VLA) make this correction in the real-time
system. This effect is important if the distance to the object is comparable
to or less than:
\begin{equation}
{B^2\over\lambda}
\end{equation}
where B is the baseline. Note that the sign of the correction could be
in error in this experimental version: try using a negative distance
as well as a positive distance.
\end{description}

\subsubsection*{What {\tt imager} needs:}

{\tt imager} operates on a specified MeasurementSet to produce any of a
range of different types of image: dirty, point spread function,
clean, residual, {\em etc.}  A MeasurementSet is the holder for
measurements from a telescope. It is simply an \aipspp\ Table obeying
certain conventions as to required and optional contents. The
intention is that it should contain all the information needed to
reduce synthesis and single dish observations (see
\htmladdnormallink{\aipspp\ Note 191} {../../notes/191.ps}). A
UVFITS file can be converted to a MeasurementSet using the
\ahlink{fitstoms}{ms:ms.fitstoms.constructor} tool function (a constructor of the
\ahlink{ms}{ms} tool.

{\tt imager} adds some extra columns to the MeasurementSet to store
results of processing. The following columns in the MS are
particularly important:
\begin{description}
\item[DATA] The original observed visibilities are in a column
called DATA. These are not altered by any processing in \aipspp.
\item[CORRECTED\_DATA] During a calibration process, as carried out by
{\em e.g.} \ahlink{calibrater}{calibrater}, the visibilities may be corrected for
calibration effects. This corrected visibilities are stored in a column
CORRECTED\_DATA which is created on demand by calibrater and imager. In creating
the CORRECTED\_DATA column, {\tt imager} will only correct for parallactic
angle rotation. This can be controlled using the
\ahlink{correct}{correct} tool function. All imaging performed by
imager is from the CORRECTED\_DATA column (apart from the
tool function \ahlink{makeimage}{imager:imager.makeimage} which can also make dirty
images from the other visibility columns).
\item[MODEL\_DATA] During various phases of processing, the
visibilities as predicted from some model are required. These 
model visibilities are stored in a column MODEL\_DATA. 
These are used by the \ahlink{calibrater}{calibrater} tool for 
calibration.
\item[IMAGING\_WEIGHT] Weighting of data (including natural,
uniform and Briggs weighting,
and tapering) is accomplished by setting the column IMAGING\_WEIGHT
appropriately. 
\end{description}
Standard tools such as the \ahlink{table}{table}
module and the \ahlink{ms}{ms} can be used to access and possibly
change these (and all other) columns.

{\tt imager} can handle an initial model in a number of forms: as an
image, as a list of images, as a \ahlink{componentmodels:componentlist}
{componentmodels:componentlist}, or as some combination. Fitting of
{\tt componentmodels} is planned but is not currently supported.

{\tt imager} uses a number of scratch files. Following \aipspp\
practice, these are placed in the directories specified in the
aipsrc variable user.directories.work.
Those disks that possess sufficient free disk space are
chosen in sequence. So to spread your scratch files over
two disks each of which has a directory tcornwel/tmp do {\em
e.g.}

\begin{verbatim}
user.directories.work:   /bigdisk1/tcornwel/tmp /bigdisk2/tcornwel/tmp
\end{verbatim}

\subsubsection*{How to control imager:}

To use {\tt imager}, one has to construct a {\tt imager} tool using
a MeasurementSet as an argument, for example:

\begin{verbatim}
myimager:=imager('3C273XC1.ms')
\end{verbatim}

The Glish variable {\tt myimager} then contains the tool functions
that may be used to do various operations on the MeasurementSet
3C273XC1.ms. These tool functions can be broken down into those that
set {\tt imager} up in some state, and those that actually do some
processing. The setup tool functions are: 

\begin{description}
\item[setimage] is a {\em required} tool function that defines
the parameters (size, sampling, phase center, {\em etc.}) of
any image that is to be constructed. If you omit to call
setimage prior to any operation that needs these parameters,
an error message will result. setimage is passive: nothing 
happens immediately but subsequent processing is altered.
\item[setdata] is an {\em optional} tool function that selects which data
are to be operated on during the processing. This selection can
consist of choosing the spectral windows or fields that are
to be operated on, or setting channels that are to be operated
on in subsequent processing. setdata is active: the selection
occurs immediately and is effective for all subsequent operations
(until setdata is called again).
\item[setoptions] is an {\em optional} tool function that sets parameters
of lesser importance such as gridding parameters, cache sizes. While
these affect the processing, usually the default values will
suffice. setoptions is passive: nothing happens immediately but
subsequent processing is altered.
\item[setbeam] is an {\em optional} tool function that sets the
parameters of the synthesized beam to be used in restoring deconvolved
images. setbeam is passive: nothing happens immediately but
subsequent processing is altered.
\item[setvp] is an {\em optional} tool function that sets the
parameters of the voltage pattern model used in mosaicing.
setvp is passive: nothing happens immediately but
subsequent processing is altered.
\item[setsdoptions] is an {\em optional} tool function that sets the
relative scaling and weighting of single dish data versus 
interferometer data and also other single dish specific parameters like the convolution support when doing single dish imaging.
\end{description}

Thus to understand what imager is doing, one has to remember that at
any time, it has a {\em state} that has been set by using these
tool functions. The state may be viewed in one of two ways: either 
\ahlink{summary}{imager:imager.summary} can be used to output the
current state to the logger, or, in the GUI, the current
state of these parameters is displayed and updated following any
relevant changes. 

All the other tool functions of {\tt imager} are active: something happens
immediately. Hence, for example, the \ahlink{weight}{imager:imager.weight}
tool function acts immediately to change the weighting of the selected
data. In particular, unlike other packages, it does {\em not} set the
weighting parameters for latter operations. The
\ahlink{clean}{imager:imager.clean} tool function performs a clean deconvolution of
an image, reading and writing a model image. Note that operations that
require or produce an image usually take an appropriate image name in
the argument list.  Often if such an image is not given then it is
constructed using the image parameters set via
\ahlink{setimage}{imager:imager.setimage} and using an appropriate name
({\em e.g.} a restored image is named from the model image by
appending .restored so that 3C273XC1.clean becomes
3C273XC1.clean.restored).

The concept of the {\em state} of imager bears a little more
explanation.  The MeasurementSet can potentially contain data for many
different fields and spectral windows. One therefore has to have some
way of distinguishing which data are to be included in
processing. Rather than have each possible tool function ({\em e.g.}
weight, image, clean) take a long list of parameters to determine
which data are to be included, {\tt imager} has a
\ahlink{setdata}{imager:imager.setdata} tool function that sets {\tt
imager} up some that in subsequent processing only the selected data
are processed. For example, to select only field id 1 and spectral
windows 1 and 2, one would do:

\begin{verbatim}
myimager.setdata(fieldid=1, spwid=1:2)
\end{verbatim}

The state of {\tt imager} also consists of information about the
default image settings (set via \ahlink{setimage}{imager:imager.setimage})
and various less important options (set via 
\ahlink{setoptions}{imager:imager.setoptions}).

\subsubsection*{What {\tt imager} produces:}

{\tt imager} reads and writes \aipspp\ MeasurementSets and Images.  The
format of images is 4 dimensional, with the first two being right
ascension and declination, the third being polarization and the fourth
being frequency. By suitable choice of the input parameters, one can
make images of $I$ alone, $I,V$ or $I,Q,U,V$ for one or all channels.
The \ahlink{makeimage}{imager:imager.makeimage} tool function can also make a complex image
of the original polarizations {\em e.g.} RR, RL, LR, LL. This latter
type of image is useful for diagnostic purposes.

Images generated by {\tt imager} may be viewed using
\ahlink{viewer}{viewer:viewer} tool or retrieved using the
\ahlink{images}{images} tool, the MeasurementSets may be accessed
using the \ahlink{ms}{ms} tool. More on this in the example below.

\subsubsection*{What {\tt imager} does not do:}

{\tt imager} does not handle calibration of visibility data beyond correction
for parallactic angle variations.  Instead, you should use the
\ahlink{calibrater}{calibrater} tool for this purpose.  However, {\tt imager} and calibrater can
cooperate on the self-calibration of data. 
%using the tool function
%\ahlink{imager.selfcal}{imager:imager.selfcal}.

\subsubsection*{What improvement to {\tt imager} are in the works:}

We are currently working on a number of improvements:

\begin{itemize}
\item Improved gridder to handle many telescopes and many channels
more efficiently.
\item Parallelized CLEAN and gridding
\end{itemize}

\subsubsection*{Advanced use of imager:}

As with all \aipspp\ applications, {\tt imager} is designed to be open: all
the results are written to and read from standard \aipspp\ table files.
This open design of {\tt imager} also allows the user to try out new
methods of processing data.  Models may be read into Glish, edited or
manipulated via standard Glish facilities, and then written out and
used subsequently in {\tt imager}.  Suppose that we want to halve the
Stokes I of all pixels with negative Stokes I. The following Glish
fragment does the trick:
\begin{verbatim}
m:=image('myimage')
shape:=im.shape()
blc:=[1,1,1,1]
trc:=[shape[1],shape[2],1,shape[4]]
a:=m.getchunk(blc,trc)
a[a<0.]*:=0.5
m.putchunk(a,blc)
m.flush()
m.close()
\end{verbatim}

\protect\subsubsection*{Overview of {\tt imager} tool functions:}

\protect\begin{description}
\protect\item[Data access] \ahlink{open}{imager:imager.open},
\ahlink{close}{imager:imager.close}, \ahlink{done}{imager:imager.done}
\protect\item[Data selection] \ahlink{setdata}{imager:imager.setdata}
\protect\item[Data editing] \ahlink{clipvis}{imager:imager.clipvis}
\protect\item[Data calibration] \ahlink{correct}{imager:imager.correct}
\protect\item[Data examination] \ahlink{plotvis}{imager:imager.plotvis}, \ahlink{plotuv}{imager:imager.plotuv}, 
\ahlink{plotweights}{imager:imager.plotweights}, \ahlink{plotsummary}{imager:imager.plotsummary}
\protect\item[Weighting] \ahlink{weight}{imager:imager.weight}, \ahlink{filter}{imager:imager.filter}, 
\ahlink{uvrange}{imager:imager.uvrange}, \ahlink{sensitivity}{imager:imager.sensitivity}, 
\ahlink{fitpsf}{imager:imager.fitpsf}, \ahlink{plotweights}{imager:imager.plotweights}
\protect\item[Image definition] \ahlink{advise}{imager:imager.advise}, \ahlink{setimage}{imager:imager.setimage}, \ahlink{make}{imager:imager.make}
\protect\item[Imaging] \ahlink{makeimage}{imager:imager.makeimage},
\ahlink{clean}{imager:imager.clean},
\ahlink{nnls}{imager:imager.nnls}, \ahlink{mem}{imager:imager.mem}, 
\ahlink{pixon}{imager:imager.pixon}, 
\ahlink{restore}{imager:imager.restore},
\ahlink{residual}{imager:imager.residual}, 
\ahlink{approximatepsf}{imager:imager.approximatepsf}, 
\ahlink{fitpsf}{imager:imager.fitpsf}, 
\ahlink{setbeam}{imager:imager.setbeam}, 
\ahlink{ft}{imager:imager.ft}, 
\ahlink{smooth}{imager:imager.smooth},
\ahlink{feather}{imager:imager.feather},
\ahlink{makemodelfromsd}{imager:imager.makemodelfromsd}
\protect\item[Masks] \ahlink{mask}{imager:imager.mask},
\ahlink{boxmask}{imager:imager.boxmask},
\ahlink{regionmask}{imager:imager.regionmask},
\ahlink{exprmask}{imager:imager.exprmask},
\ahlink{clipimage}{imager:imager.clipimage}
\protect\item[Miscellaneous] \protect\ahlink{summary}{imager:imager.summary}, 
\protect\ahlink{setoptions}{imager:imager.setoptions},
\protect\ahlink{setoptions}{imager:imager.setmfcontrol},
\protect\ahlink{setoptions}{imager:imager.setsdoptions}
\end{description}



\begin{ahexample}
The following example shows the quickest way to make a CLEAN image and
display it. Note that this can be more easily done from the
\ahlink{toolmanager}{tasking:toolmanager}.
\begin{verbatim}
include 'imager.g'
#
# First make the MS from a FITS file:
#
m:=fitstoms(msfile='3C273XC1.MS', fitsfile='3C273XC1.FITS'); m.close();
#
# Now make an imager tool for the MS
#
imgr:=imager('3C273XC1.MS')      
#
# Set the imager to produce images of cellsize 0.7 and
# 256 by 256 pixels
#
imgr.setimage(nx=256,ny=256, cellx='0.7arcsec',celly='0.7arcsec');
#
# Wait for results before proceeding to the next step
#
dowait:=T
#
# Make and display a clean image
#
imgr.clean(niter=1000, threshold='30mJy',
model='3C273XC1.clean.model', image='3C273XC1.clean.image')
dd.image('3C273XC1.clean.image')
#
# Fourier transform the model 
#
imgr.ft(model='3C273XC1.clean.model')
#
# Plot the visibilities
#
imgr.plotvis()
#
# Write out the final MS and close the imager tool
#
imgr.close()
\end{verbatim}
\end{ahexample}
\input{imager.htex}
\input{vpmanager.htex}
\end{ahmodule}
