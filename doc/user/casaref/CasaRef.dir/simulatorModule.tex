%% Copyright (C) 1999,2000,2001,2003
%% Associated Universities, Inc. Washington DC, USA.
%%
%% This library is free software; you can redistribute it and/or modify it
%% under the terms of the GNU Library General Public License as published by
%% the Free Software Foundation; either version 2 of the License, or (at your
%% option) any later version.
%%
%% This library is distributed in the hope that it will be useful, but WITHOUT
%% ANY WARRANTY; without even the implied warranty of MERCHANTABILITY or
%% FITNESS FOR A PARTICULAR PURPOSE.  See the GNU Library General Public
%% License for more details.
%%
%% You should have received a copy of the GNU Library General Public License
%% along with this library; if not, write to the Free Software Foundation,
%% Inc., 675 Massachusetts Ave, Cambridge, MA 02139, USA.
%%
%% Correspondence concerning AIPS++ should be addressed as follows:
%%        Internet email: aips2-request@nrao.edu.
%%        Postal address: AIPS++ Project Office
%%                        National Radio Astronomy Observatory
%%                        520 Edgemont Road
%%                        Charlottesville, VA 22903-2475 USA
%%
%% $Id$

\begin{ahmodule}{simulator}{Module for simulation of telescope data}

\ahcategory{aips}

\bigskip
\noindent{\bf Description} 

simulator provides a unified interface for simulation of telescope
processing. It can create a MeasurementSet from scratch or read in
an existing MeasurementSet. It can predict synthesis data onto the
(u,v) coordinates or single dish data onto (ra,dec) points, and it 
can corrupt this data through Gaussian errors or through specified
errors reading in (anti-) calibration tables.

In the observing phase, simulator tries to act like a (simple)
telescope. You first open the name of the MeasurementSet that
you wish to construct. Next you use the various set* methods
to setup the observing (sources, spectral windows, etc). Each
such setup should be given a unique name that will be used in
the next step. Then you call the observe method for each observing scan 
you wish to make. Here you specify the source name, spectral window 
name, and observing times. After this, you have a MeasurementSet that is
complete but empty. In the next phase, you fill the
MeasurementSet with data from a model and then corrupt the measurements
(if desired). To fill it in with a model, use the predict method.
Finally, to apply errors, first set up the various effects using the 
relevant set* methods and then call corrupt.

Some important points:
\begin{itemize}
\item One call to {\it observe} generates one scan (all rows have the same SCAN\_NUMBER).
\item The start and stop times specified to {\it observe} need not be contiguous and so
one can siulate antenna drive times.
\item Currently there is no facility for patterns of observing, such as mosaicing, since it is easy to
do this via sequences of calls of observe.
\end{itemize}

The following columns of the MeasurementSet are particularly important:
\begin{itemize}
\item {\bf DATA} The original observed visibilities are in a column called DATA. These are normally not altered by any processing in CASA. {\bf simulator} does write this column when it creates observations.
\item {\bf CORRECTED\_DATA} During a calibration process, the visibilities may be corrected for calibration effects. These corrected visibilities are stored in CORRECTED\_DATA which is created upon demand.
\item {\bf MODEL\_DATA} During various phases of processing, the visibilities as predicted from some model are required. These model visibilities are stored in the column MODEL\_DATA. The 'ft' task can be used to calculate the model visibilitiy for a model image.
\end{itemize}

\noindent The available tool in this module is:

\begin{itemize}

\item \ahlink{Simulator}{simulator:simulator} - tool for simulation
\end{itemize}


\ahobjs{}
\ahfuncs{}

\input{simulator.htex}

\end{ahmodule}
